%------------------------------------------------
%            Import packages
%------------------------------------------------
\usepackage[english]{babel}
\usepackage[utf8x]{inputenc}
%  Allow nice numbering of lists.
%\usepackage{enumitem}
\usepackage{enumerate}
%  Prevent cutting of paragraphs.
\usepackage[defaultlines=6,all]{nowidow}
%  Add some math
\usepackage{amsmath}
\usepackage{amsthm}
%  Add some graphics
\usepackage{graphicx}
%  Add the possibility of adding To-Do-Notes
\usepackage[colorinlistoftodos,obeyFinal]{todonotes}
%  Get color and href package
\usepackage{color}
\usepackage{xcolor}
\usepackage{hyperref}
%  Let's do some calculations
\usepackage{fp}
%  Set the color of the links
\hypersetup{
	colorlinks=true,
	linkcolor=red,
	filecolor=red,
	urlcolor=blue,
}
% Tikz to draw some things.
\usepackage{tikz}
\usetikzlibrary{arrows,positioning, calc}
% To keep track of our time spend on this project.
\usepackage{logbook}
% Helps positioning the images.
\usepackage{float}
% To show some source code.
\usepackage{listings}
\definecolor{pblue}{rgb}{0.13,0.13,1}
\definecolor{pgreen}{rgb}{0,0.5,0}
\definecolor{pred}{rgb}{0.9,0,0}
\definecolor{pgrey}{rgb}{0.46,0.45,0.48}

\usepackage{listings}
\lstset{
	language=Java,
    showspaces=false,
    showtabs=false,
    keepspaces=true,
    tabsize=2,
    breaklines=true,
    showstringspaces=false,
    breakatwhitespace=true,
    commentstyle=\color{pgreen},
    keywordstyle=\color{pblue},
    stringstyle=\color{pred},
    basicstyle=\ttfamily,
    moredelim=[il][\textcolor{pgrey}]{$$},
    moredelim=[is][\textcolor{pgrey}]{\%\%}{\%\%},
    numbers=left,
    numbersep=5pt,
    numberstyle=\tiny\color{pgrey},
    captionpos=t,
    frame=single,
    title=\lstname
}

% You might need to make an apendix, this will make that a lot easier.
\usepackage[toc,page]{appendix}

% Let's add some comments.
\usepackage{comment}

\setlength{\parindent}{0pt} % Prevents indention of all paragraphs.

%------------------------------------------------
%            Define folder locations.
%------------------------------------------------
\newcommand{\final}{final/}
\newcommand{\reflection}{reflection/}
\newcommand{\orientation}{orientation-phase/} % Set the folder of the orientation phase.
\newcommand{\machineDesign}{machine-design-phase/} % Set the folder of the machine design phase.
\newcommand{\softwareDesign}{software-design-phase/} % Set the folder of the software design phase.
\newcommand{\softwareSpecification}{software-specification-phase/} % Set the folder of the software design phase.
\newcommand{\softwareImplementation}{software-implementation-phase/} % Set the folder of the software implementation and integration phase.
\newcommand{\systemValidation}{system-validation-phase/} % Set the folder of the system validation and verification phase.
\newcommand{\logbook}{logbook/} % Set the folder of the orientation phase.
\newcommand{\image}{images/} % Set the folder of the images.
\newcommand{\img}{\image}
\newcommand{\src}{source-code/}
\newcommand{\appendixFolder}{appendix/}
\newcommand{\pseudocode}{pseudo-code/}
\newcommand{\agenda}{agenda/}

%------------------------------------------------
%            Add reference commands
%------------------------------------------------
% One single link
\newcommand*{\fullref}[1]{\hyperref[{#1}]{\autoref*{#1} \nameref*{#1}}}

% Reference in the form: "(<title>, <type> <number>)", so for example "(Structure, section 6.3)".
\newcommand*{\longref}[1]{\hyperref[{#1}]{(\nameref*{#1}, \autoref*{#1})}}

% Reference for appendix.
\newcommand*{\aref}[1]{\hyperref[#1]{Appendix \ref{#1}: \nameref*{#1}}}

% Creates a reference to a (sub)section/chapter like this "Requirements (see subsection 3.2)".
% @input	Name of the label to refer to.
% @output	Generates a title in the format "$Title (see $Type$ $Number$)".
\newcommand*{\seeref}[1]{\hyperref[{#1}]{\nameref*{#1} (see \autoref{#1})}} 

%------------------------------------------------
%            All other commands
%------------------------------------------------
% Add an email address with link.
% @input	The email address which should be added.
% @output	The email address surrounded with a link.
\newcommand{\email}[1]{\href{mailto:#1}{#1}}

% Sets the title in the correct way.
% @input	Title of the document
% @input	Version number
% @output	A correctly set title.
\newcommand{\settitle}[2]{\title{{#1} \\ OGO 2IO70 \\ Version {#2}}}

% Sets the author of the file in the correct way.
% @input	The full name of the author.
% @input	The email address of the author.
\newcommand{\setauthor}[2]{\author{Group 2 \\ {#1} \\ \email{#2}}}

% Makes text bold.
% @input	The text which should be displayed bold.
% @output	The text made bold.
\newcommand{\bold}[1]{\textbf{#1}}
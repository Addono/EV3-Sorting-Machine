\section{Preface}
In the project guide it was stated that we should construct a computer program that realizes the functions specified in the Software Specification deliverable during this phase/in this deliverable. The program should be written in (elementary) Java and does not even have to be compilable and executable. The idea is that this program written in elementary Java would serve as a stepping stone towards the Assembly Language program for the PP2 written in the next phase. The thing is that we are on of the LEGO MindStorm Trial groups. We are given a little bit more freedom of choice. Instead of the PP2 processor, we are given the EV3 brick (like mentioned before), and instead of the Fisher Technik kit, we are given the LEGO MindStorm kit. \\

In section 3.1.3 we have seen the various programming languages options for this EV3 brick. The EV3 comes by default with its own LEGO firmware and software. We have \emph{chosen} to flash other firmware onto the brick so that we can make use of leJOS (instead of the drag-and-drop interface that comes with the official firmware). \emph{The reason we have chosen} to program the sorting-machine in leJOS is because leJOS is Java based, and this gives us more options and freedom programming wise and allows more complex programming structures. This is however a little bit in conflict with the  description of contents of the software design deliverable as stated in the project guide. Since it is the case that with the firmware replacement leJOS a Java virtual machine is included, we are able to program in Java. There are leJOS plugins for some Java IDE's like NetBeans and Eclipse (we are going to use the latter one). This is the reason of the conflict. The project guide states that during this phase the program should be written in (elementary) Java, \emph{but since our actual source code for the sorting machine (see Software Implementation and Integration)} is written in Java, it seems a little bit overkill to write the program first in elementary Java during this phase, and to write the source code in Java again during the next phase. We have \emph{therefore chosen that we} are not going to write the source code first in elementary Java, but to do it in pseudo-code (checked with the tutor first if this was ok). This will then serve as a stepping stone to the actual source code, written in Java.\\

Since the implementation of our program is so close to the actual finite state automaton, the correctness of the program is mostly proven  by the correctness of the finite automaton which we have seen in software specification and will be further discussed in the section System Validation and Testing.
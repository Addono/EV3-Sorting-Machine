We have worked hard for several weeks, building and documenting our sorting machine. This was our first group project of this magnitude. Although our project may not have any money value, we still learned a lot these past weeks.\\

we learned how to
\begin{itemize}
\item plan these kind of group projects in a work plan and its added benefits.
\item structurally work over the course of our entire project using the V-model.
\item construct and share code together using Git and GitHub.
\item improve our presentation skills.
\item document these kind of projects.
\item be creative with machine design to handle lack of components.
\item work with the EV3 Brick and its associated motor and sensors.
\end{itemize}


Along the way we encountered some problems we had to overcome.


Somewhere during the early stages of our project, we were confronted with the fact that one of our group members was unfortunately no longer going to participate in this project. That left us us with 5 members for a project originally meant for 6. We solved this by adapting the work plan, dividing the workload among the 5 of us.


While working with the color sensor we encountered the problem that it can't reliably tell some of the LEGO components apart form the disks. We solved this by replacing those LEGO parts in our machine design with differently colored parts.


We encountered 2 problems regarding disk insertion.
The first problem was that We weren't given a tube to insert our disks. We solved this problem by building a tube ourselves that was long enough to hold all 12 discs.
The second problem was that we wanted to count the amount of disks that got inserted, but the pressure sensor isn't all that sensitive. We solved this problem by decreasing the height of the insertion point of the tube so that disks need to be pushed passed the sensor.


While working on the construction of the sorting wheel we encountered several problems. problems with telling the colors of the wheel apart form the disks, the wheel getting stuck, the disks sliding out and recognizing the teeth for calibration. After a lot of reconstructing and adapting we solve these problems. We ended up with a wheel constructed with red LEGO parts, of which the sensor could tell apart close by and further away, with teeth of 2.5 LEGO pieces high and rubber bars below the wheel it to prevent disks from sliding past it.


We ran into some problems while getting the EV3 to run self written code in general, since we didn't want to use the drag and drop system. We solve this problem by buying a SD card for the EV3 brick and flashing LEJOS Software on it.

In the end it when the machine was working properly and it could even sort 12 disks in under 6 seconds, we were very proud of what we had actually build and programmed. This is one of the most fun courses we have had up till now. We would like to thank Pieter Cuijpers for giving us the opportunity to be one of the (LEGO EV3) trial groups and we would like to thank Alberto Corvo for guiding and helping us along the road. It is all truly appreciated. 



